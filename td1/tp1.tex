\documentclass{article}
\usepackage{titlepage}
\graphicspath{{image/}}

\title{TD1}
\subtitle{HTML, CSS et XMLHttpRequest}
\dateremise{le 30 Janvier 2018}
\author{Jacob Dorais}{Billy Bouchard}{B1}
\prof{Tanga Mathieu Kabor\'e}
\begin{document}
\maketitle

\section{Couleur}
\par
 Lors de la sélection de notre palette de couleur, nous avons opté pour un thème plutôt foncé, afin d'accoutumer les utilisateurs et développeurs qui consulteraient le site web à une heure du soir où les yeux sont plus sensibles à la lumière. L’orange et jaune ressort bien du dégradé linéaire grisâtre en arrière-plan, en plus d'être des couleurs chaudes, ce qui rend notre site web plus confortable. En jaune, nous avons les éléments clés, tels les formules, le bouton ainsi que les bordures du tableau pour que l'information soit bien encadrée. Pour cette même raison, les rangées impaires ont un ton de gris légèrement différent des rangées paires.
\section{Tableau}

\par

Pour enchaîner sur nos tableaux, nous avons laissé les bordures carrées afin de garder un look professionnel. Comme demandé, nous n'affichons que les éléments qui ont une concentration plus élevée que 0.0001. Nous offrons toutefois la possibilité d'afficher l'entièreté des données à l'aide du bouton ''show all''. comme 4 éléments sont des conditions, nous avons une colonne Conditions qui contient les 4 colones Temperature, pression, type et A. De cette manière, notre tableau est ordonné et facile à lire.

\subsection{Bouton}

\par

Le bouton a les coins arrondis et possède une ombre de 7px en dessous, ce qui donne une impression 3 dimensions. Lorsque nous pressons le bouton, une translation vers le bas ainsi qu'une diminution de l'ombre donne l'impression d'un vrai bouton qui est enfoncé. Lorsqu'on passe le curseur au-dessus du bouton, la couleur devient plus sombre, l'utilisateur sait maintenant qu'il est en position pour cliquer sur le bouton. Donner l'impression que le bouton est réel est utile pour rendre notre site web intuitif, l'utilisateur normal va mieux percevoir le bouton et le comprendra plus facilement.

\section{Espace}

\par

Pour l'espacement général de nos paragraphes, nous avons centré le texte et ajouté des marges à gauche ainsi qu'à droite pour augmenter la lisibilité. Nous avons remarqué que lors de la lecture d'un texte écrit de la gauche de l'écran à droite, il devient difficile de sauter de ligne sans se tromper. Il était donc préférable d'avoir le corps du site web avec une largeur de 60 pour cent.

\end{document}

